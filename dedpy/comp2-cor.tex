% Created 2009-11-17 Tue 22:35
\documentclass[11pt]{article}
\usepackage[utf8]{inputenc}
\usepackage[T1]{fontenc}
\usepackage{graphicx}
\usepackage{longtable}
\usepackage{float}
\usepackage{wrapfig}
\usepackage{soul}
\usepackage{amssymb}
\usepackage{hyperref}
\usepackage{setspace}

\title{Rosamunda}
\author{Dan Davison}
\date{17 November 2009}

\begin{document}

\maketitle


\doublespace

En el cuento de Carmen Laforet, después de despertarse en un tren,
``Rosamunda'' conoce a un soldado y le cuenta una versión de la historia
de su vida. Aunque no solicitó la información, el soldado la escucha
con atención, hasta que finalmente la convida a desayunar. La
situación no merecería ser registrada si no hubiera por lo menos tres
elementos atípicos. Primero, el encuentro es entre una mujer en las
últimas décadas de su vida, y un hombre en las primeras décadas de su
vida. Segundo, el comportamiento de la mujer, y su aspecto físico,
revelan una persona que no sigue las normas de sociedad, y además que
está obsesionada por su vida anterior. Y tercero, la historia que
Rosamunda cuenta es extremamente dudosa.

Consideremos primero la historia contada por Rosamunda. Según ella, se
casó con dieciséis años -- ya famosa por sus talentos artísticos -- y
se encontró muchos años después, casada, con hijos y un esposo que
poseía cada característica no deseable en un esposo. Después de la
muerte del único hijo con el cual se llevó bien, dejó su familia y
reanudó su carrera artística, siendo ``aclamada de nuevo por el
público''. Ahora se encuentra, después de (según ella) haber repartido
su fortuna entre los pobres, volviendo a su esposo porque él ``no puede
vivir'' sin ella.

El narrador confirma que varios elementos de esa historia son
productos de su imaginación. Lo que parece ser verdad es que al volver
a la ciudad, pasó sus días en los vestidos de su juventud, sin casa ni
empleo. No es muy sorprendente que desease la atención del joven
soldado: había perdido cualquier sentido de las normas de
comportamiento, y además estaba obsesionada por la memoria de su
juventud.

Lo que quizás sea más interesante es el motivo del soldado. No era el
único soldado en el tren, y podemos imaginar que, en un viaje largo y
tedioso, conocer a alguna mujer joven, y desayunar con ella, sería
considerado por los demás como un éxito, aunque pequeño. Pero una loca
de cuarenta años mas vieja? Por agradable que sea, no es una acción
típicamente viril, y podemos inferir que el soldado joven
probablemente, como Rosamunda, no ocupa una posición central en la
estructura social. Pero sus motivos siguen siendo misteriosos. Cuando
está imaginando las ``ovaciones delirantes y su propia figura
\ldots{} recibiéndolas'', es como si creyera su historia, 


\section*{[Desde este punto, no había acceso a la versión corregida]}
\label{sec-1}



y en ese momento parece estar motivado por una fantasía romántica con
respecto a la mujer más vieja. Sin embargo, después, el narrador
revele pensamientos suyos que no están tan ingenuo: ``No cabía duda de
que estaba loca, la pobre.'' En convidarla a desayunar, el soldado
parece tener el idea de que impactaría sus amigos: ``Y si contara a sus
amigos que había encontrado en el tren una mujer estupenda y que\ldots{}?''
Lo que aparentemente ha olvidado es que, faltando una mujer estupenda,
sería más fácil realizar ese idea sin desayunar con nadie.

En resumen, romántico ingenuo o cínico manipulador, el soldado, como
Rosamunda, tiene una comprensión incompleta de la realidad y de la
causalidad; por muy diferentes que inicialmente parecen, los dos están
unidos por eso, y por una cierta excentricidad.


\end{document}
